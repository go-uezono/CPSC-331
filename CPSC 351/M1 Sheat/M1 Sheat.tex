\documentclass{article} 
%Packages%
\usepackage{geometry}
\geometry{a4paper,total={200mm,257mm},left=5mm,top=5mm}
\usepackage{titlesec}
\usepackage[english]{babel}
\usepackage[autostyle, english = american]{csquotes}
\usepackage{graphicx}
\usepackage{nopageno}
\MakeOuterQuote{"}

%Formatting%
\titleformat*{\section}{\bfseries\large}
\titleformat*{\subsection}{\bfseries\normalsize}

\begin{document}


\section*{RegEx}
Format:
\begin{enumerate}
    \item $a\in\Sigma$, "a" is a RegEx over $\Sigma$, where L(a) = \{a\}
    \item $b\in\Sigma$, "b" is a RegEx over $\Sigma$, where L(b) = \{b\}
    \item $c\in\Sigma$, "c" is a RegEx over $\Sigma$, where L(c) = \{c\}
    \item $\lambda\in\Sigma$, "$\lambda$" is a RegEx over $\Sigma$, where L($\lambda$) = \{$\lambda$\} 
    \item "$\Sigma$" is a RegEx, L($\Sigma$) = $\Sigma$ = \{a, b, c\} (of length 1, so $\Sigma\circ\Sigma$ has length 2)
    \item "$\Sigma$" is a RegEx (line 8), so "$(\Sigma)^\star$" is a  a RegEx over $\Sigma$, L(($\Sigma)^\star$) = $\Sigma^\star$, set of all strings
    \item "b" and "a" are RegEx (lines 2, 1), so "(b $\cup$ a)" is a RegEx over $\Sigma$, L(b $\cup$ a) = L(b) $\cup$ L(a)
    \item "(b $\cup$ a)" is a RegEx (line 4), so "((b $\cup$ a))$^\star$" is a RegEx over $\Sigma$, L(((b $\cup$ a))$^\star$) = \{b,c\}$^\star$, set of all strings
    that do not include an "a"
    \item "a" and "c" are RegEx (lines 1, 3), so "(a $\circ$ c)" is a RegEx over $\Sigma$, L(a $\circ$ c) = L(a) $\circ$ L(b), set of size one
    \item "((b $\cup$ a))$^\star$" and "(a $\circ$ c)" are RegEx (lines 5, 4), so "(((b $\cup$ a))$^\star\circ$(a $\circ$ c))" is a RegEx over $\Sigma$, \\
    L(((b $\cup$ a))$^\star$) $\circ$ L(a $\circ$ c), set of strings where any of "b" or "a" come before the last "a", followed by a "c"
    \item "c" and "$\lambda$" are RegEx (line 3, 4), so "($c\cup\lambda$)" is a RegEx over $\Sigma$, L($c\cup\lambda$) = L(c) $\cup$ L($\lambda$), set of strings that
    include at most 1 (or no) "c's"
\end{enumerate}

\section*{DFA}
To recognize this language, the DFA needs to remember cases to hold $\omega\in\Sigma^\star$: \\
List cases (e.g. \{$S_0=\omega\in\Sigma\star|\omega$ does not include an "a's"\}; corresponds to state $q_0$)
\subsection*{Sanity Check 1: If it has a finite number of subsets}
List subsets (e.g. $S_0, S_{od},S_{ev}$)
\subsection*{Sanity Check 2: Every string belongs to \textbf{exactly one} subset}
e.g. Every string must include some fixed number of "a's". Zero $(\omega\in S_0)$, odd $(\omega\in S_{od})$, or at least two and even $(\omega\in S_{ev})$.
Cannot be in more than one of these sets simultaneously: $S_0\cap S_{od}=S_0\cap S_{ev}=S_{od}\cap S_{ev}=\emptyset$
\subsection*{Sanity Check 3: If states are described as above\dots}
e.g. Then, $S_0\cap L=S_{od}\cap L=\emptyset$, while $S_{ev}\subseteq L$. From this, it follows that $q_{ev}$ is an accepting state, 
hence, $q_{ev}\in F$
\subsection*{Sanity Check 4: Transitions must be well-defined}
e.g. Transitions out of each are as follows:
\begin{itemize}
    \item Number of "b's" and "c's" don't change the number of "a's" we need:
    \begin{itemize}
        \item $\{\omega\cdot b|\omega\in S_0\}\subseteq S_0$, $\{\omega\cdot c|\omega\in S_0\}\subseteq S_0$
        \item (Same idea for sets $S_{od},S_{ev}$)
    \end{itemize}
    Thus, retained in the same state.
    \item Number of "a's" change when you add an "a":
    \begin{itemize}
        \item Zero to one: $\{\omega\cdot a|\omega\in S_0\}\subseteq S_{od}$, well-defined transition from $q_0\rightarrow q_{od}$ for "a"
        \item One to at least two: $\{\omega\cdot a|\omega\in S_{od}\}\subseteq S_{ev}$, well-defined transition from $q_{od}\rightarrow q_{ev}$ for "a"
        \item Even number to odd that is at least two: $\{\omega\cdot a|\omega\in S_{ev}\}\subseteq S_{od}$, well-defined transition from $q_{ev}\rightarrow q_{od}$ for "a"
    \end{itemize}
\end{itemize}
All sanity checks passed, we know that $S_0$ is the same as $\{\omega\in\Sigma^\star|\delta^\star(q_0,\omega)=q_0\}$ (Same for others), where $S_0,\dots$ are
the sets of strings described at the beginning of this answer. Since $F=\{q_{ev}\}, L=S_{ev}$ is the language of this DFA - as desired.

\section*{DFA NFA Equivalence}
Format:
\begin{enumerate}
    \item 1st state introduced: initally, $\hat{Q}=\emptyset$, $\hat{R}=\{\hat{q_0}\}$, so $\hat{q_0}$ is $Cl\lambda(q_0)=\{q_0\}$. \\
    Our current DFA: (Drawing of DFA w/ start state) \\
    We pick $\hat{q_0}$ since it's the only state in $\hat{R}$ so far:
    \begin{enumerate}
        \item Transitions out for $\omega=a$ for $\hat{q_0}$ \\
        $\bigcup_{r\in S_0}\bigcup_{s\in\delta(r,a)}Cl\lambda(s)=\bigcup_{r\in \{q_0\}}\bigcup_{s\in\delta(r,a)}Cl\lambda(s)$ \\
        $=\bigcup_{s\in\delta(q_0,a)}Cl\lambda(s)$ \\
        $=\bigcup_{s\in\{q_0\}}Cl\lambda(s)$ \\
        $=Cl\lambda(q_0)$ \\
        $=\{q_0\} = S_0$ \\
        Same as $S_0$, corresponding to state $\hat{q_0}$. Thus, $\hat{\delta}(\hat{q_0},a)=\hat{q_0}$
        \item Transitions out for $\omega=b$ for $\hat{q_0}$ (Same outcome as $\omega=a$)
        \item Transitions out for $\omega=c$ for $\hat{q_0}$ \\
        $\bigcup_{r\in S_0}\bigcup_{s\in\delta(r,c)}Cl\lambda(s)=\bigcup_{r\in \{q_0\}}\bigcup_{s\in\delta(r,c)}Cl\lambda(s)$ \\
        $=\bigcup_{s\in\delta(q_0,c)}Cl\lambda(s)$ \\
        $=\bigcup_{s\in\{q_0,q_1\}}Cl\lambda(s)$ \\
        $=Cl\lambda(q_0)\cup Cl\lambda(q_1)$ \\
        $=\{q_0\}\cup\{q_1\}=\{q_0,q_1\}$ \\
        This set does not correspond to any states in $\hat{Q}\cup\hat{R}$, so it will be a new set $S_1=\{q_0,q_1\}$ corresponding to the state
        $\hat{q_1}$.
    \end{enumerate}
    Now, $\hat{Q}=\{\hat{q_0}\},\hat{R}=\{\hat{q_1}\}$. Our DFA: (draw DFA w/ state $\hat{q_0}\rightarrow\hat{q_1}$)
    \item Since only $\hat{R}=\{\hat{q_1}\}$, set state to $\hat{q_1}$
    \begin{enumerate}
        \item Transitions out for $\omega=a$ for $\hat{q_1}\rightarrow$ same process, set $S_2=\{q_0,q_2\}$
        \item Transitions out for $\omega=b$ for $\hat{q_1}\rightarrow$ corresponds to state $\hat{q_0}$ ($b$ transition back to $\hat{q_0}$)
        \item Transitions out for $\omega=c$ for $\hat{q_1}\rightarrow$ corresponds to state $\hat{q_1}$ ($c$ transition remain in $\hat{q_1}$)
    \end{enumerate}
    Now, $\hat{Q}=\{\hat{q_0},\hat{q_1}\},\hat{R}=\{\hat{q_2}\}$. Our DFA: (draw DFA w/ state $\hat{q_2}$)
    \item Repeat until no more new states found
\end{enumerate}
\begin{itemize}
    \item Since $\hat{R}=\emptyset$, we have to find set $\hat{F}$ of accepting states:
    \begin{itemize}
        \item $S_0=\{q_0\}$ corresponds to $\hat{q_0}$, and $S_0\cap F=\{q_0\}\cap\{q_2\}=\emptyset$
        \item $S_1=\{q_0,q_1\}$ corresponds to $\hat{q_1}$, and $S_0\cap F=\{q_0,q_1\}\cap\{q_2\}=\emptyset$
        \item $S_2=\{q_0,q_2\}$ corresponds to $\hat{q_2}$, and $S_0\cap F=\{q_0,q_2\}\cap\{q_2\}=\hat{\{q_2\}}\neq\emptyset,\hat{q_2}\in\hat{F}$
        \item $S_3=\{q_0,q_1,q_2\}$ corresponds to $\hat{q_3}$, and $S_0\cap F=\{q_0,q_1,q_2\}\cap\{q_2\}=\hat{\{q_2\}}\neq\emptyset,\hat{q_3}\in\hat{F}$
    \end{itemize}
    Thus, $\hat{F}=\{\hat{q_2},\hat{q_3}\}$. Final DFA:
\end{itemize}
\end{document}
