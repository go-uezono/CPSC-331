\documentclass[10pt, 
a4paper, 
oneside, 
headinclude, footinclude, 
BCOR5mm]
{scrartcl}

% Required packages
\usepackage[
    nochapters, beramono, eulermath, pdfspacing, dottedtoc,
]{classicthesis}
\usepackage{arsclassica}
\usepackage[T1]{fontenc}
\usepackage[utf8]{inputenc}
\usepackage{graphicx}
\graphicspath{}
\usepackage{enumitem}
\usepackage{lipsum}
\usepackage{subfig}
\usepackage{amsmath,amssymb,amsthm}
\usepackage{varioref}
\usepackage{titlesec}
\usepackage{color}
\usepackage[linesnumbered, lined, ruled, commentsnumbered]{algorithm2e}

% Adjust title size
\titleformat*{\section}{\color{RoyalBlue}\LARGE}
\titleformat*{\subsection}{\color{black}\Large}
\titleformat*{\subsubsection}{\color{gray}}
\titleformat*{\paragraph}{\large}

% Theorem Styles

% Style used for definitions and examples
\theoremstyle{definition}
\newtheorem*{definition}{Definition}

% Style used for theorems, lemmas, propositions, corollaries
\theoremstyle{plain}
\newtheorem{theorem}{Theorem}[section]
\newtheorem{lemma}{Lemma}

% Style used for remarks and notes
\theoremstyle{remark}

% Set up algorithm 
\SetAlgoLined
\SetKwData{Left}{left}\SetKwData{This}{this}\SetKwData{Up}{up}
\SetKwFunction{Union}{Union}\SetKwData{FindCompress}{FindCompress}
\SetKwInOut{In}{input}\SetKwInOut{Out}{output}
\newcommand\mycommfont[1]{\footnotesize\ttfamily\textcolor{purple}{#1}}

% Hyperlinks
\hypersetup{
colorlinks=true, breaklinks=true, bookmarksnumbered, urlcolor=webbrown, 
linkcolor=RoyalBlue, citecolor=webgreen,
pdfauthor={\textcopyright},
pdfsubject={},
pdfkeywords={},
pdfcreator={pdfLaTeX},
pdfproducer={LaTeX with hyperref and ClassicThesis}
}

\hyphenation{Fortran hy-phen-ation}

%----Title and Authors----%
\title{\normalfont\spacedallcaps{CPSC 331: Data Structures, Algorithms, and their Analysis}}
\author{\spacedlowsmallcaps{Go Uezono}}

\begin{document}

%----Headers----%
\renewcommand{\sectionmark}[1]{\markright{\spacedlowsmallcaps{#1}}}
\lehead{\mbox{\llap{\small\thepage\kern1em\color{halfgray} \vline}\color{halfgray}\hspace{0.5em}\rightmark\hfil}}

\pagestyle{scrheadings}

%----Table of Contents----%
\maketitle
\setcounter{tocdepth}{2}
\tableofcontents
\listoffigures
\listoftables

\newpage
%----Algorithmic Analysis----%
\section{Algorithmic Analysis}
\subsection{Loop Invariants}
\subsection{Bound Functions}

%----Data Structures----%
\section{Data Structures}

%----Lists----%
\subsection{Lists}

%----Stacks----%
\subsection{Stacks}

%----Queues----%
\subsection{Queues}

%----Binary Search----%
\subsection{Binary Search Trees}

\newpage
%----Red and Black Trees----%
\subsection{Red and Black Trees}

%----Properties----%
\subsubsection{Properties}
A \textbf{red-black tree} is a concrete implementation of a \textbf{self-balancing binary-search tree} (reference here) that automatically maintains balance. 
Giving each node their respective color ensures that no path is more than twice as long as any other, thus is able to maintain approximate balance.\
\begin{enumerate}
    \item Every node is {\color{red}red}/black
    \item Root must be black
    \item Leaves (\textit{null}) are black
    \begin{itemize}
        \item \textit{null} vertices contain no values, while other (interior) do
    \end{itemize}
    \item If a node is {\color{red}red}, then both its children are black
    \item For each node, all simple paths from the node to descendant leaves contain the same number of black nodes 
\end{enumerate}

%----Lemma----%
The following \textbf{lemma} shows why red-black trees make good search trees:
\begin{lemma}
    A red-black tree with $n$ internal nodes has height at most $2\log (n+1)$
\end{lemma}
\begin{proof}
    Start by showing subtree rooted at any ndoe $x$ x contains at least a $2^{bh(x)}-1$ internal nodes. We prove this by \textbf{mathematical induction} on the height of $x$.
    \begin{description}
        \item [\textbf{Claim:}] If height of $x=0$, then the leaf must be \textit{T.null}, and the subtree rooted at $x$ contains at least $2^{bh(x)}-1=2^0-1=0$ internal nodes.
        \begin{description}
            \item [\textbf{Inductive step:}] 
            \begin{itemize}
                \item Consider a node $x$ that has positive height and is an internal node with two children.
                \item Each \textit{child} has a black-height of either $bh(x)$ or $bh(x)-1$ (depending on whether it is {\color{red}red} or black respectively).
                \item Since height of a \textit{child} of $x$ is less than the height of $x$ itself, we can apply the \textbf{I.H} to conlude that:
                \begin{itemize}
                    \item Each child has at least $2^{bh(x)-1}-1$ internal nodes.
                \end{itemize}
            \end{itemize}
            \item Thus, subtree rooted at $x$ contains at least $$(2^{bh(x)-1}-1)+(2^{bh(x)-1}-1)+1$$ internal nodes, which proves the claim.
        \end{description} 
        \item To complete the proof, let $h$ be the height of thr tree. According to property 4 (reference above Properties), at least half the nodes from the root 
        to a leaf (not including the root) must be black.
        \item Consequently, the $bh$ of the root must be at least $h/2$; thus, $$n \geq 2^{h/2}-1$$
        \item Moving $1$ to the left side and taking log on both sides yields: $$\log(n+1) \geq h/2$$ or $$h \leq 2\log(n+1)$$
    \end{description}
\end{proof}

\newpage
%----Rotations----%
\subsubsection{Rotational Properties}
Search operations \textit{TREE-INSERT} and \textit{TREE-DELETE} take $O(\log n)$ time. Since modifications are done to the tree, we must change the color of some of the nodes.
\begin{definition}[\textbf{Rotation}]
    Local operation that preserves the binary-tree property. 
    \begin{itemize}
        \item \textbf{Left Rotation:} assume that its right child $y$ is not \textit{null} 
        \item \textbf{Right Rotation:} assume that its left child $y$ is not \textit{null}
        \begin{itemize}
            \item $x$ can be any node on the tree whose respective child is not \textit{null}
            \item Left/Right rotations "pivots" around the link from $x$ to $y$
            \item Makes $y$ the new root, $x$ as $y$'s left(right) child, $y$'s left(right) child as $x$'s right(left) child
        \end{itemize}
        \item Both L/R rotates run in $O(1)$ time
        \item Only pointers are changed, all attributes in a node remain the same
    \end{itemize}
\end{definition}

\begin{algorithm}
    \caption{Left-Rotate($T,x$)}

    $x = y.right$ \tcp*{set y}
    $x.right = y.left$ \tcp*{Turn y's left subtree into $x$'s right subtree}
    \uIf{$y.left \neq T.null$}
        {$y.left.p = x$\;}
    $y.p = x.p$\;
    \uIf{$x.p == T.null$}
        {$T.root = y$\;}
    \uElseIf{$x == x.p.left$}
        {$x.p.left = y$\;}
    \uElse{$x.p.right = y$\;}
    $y.left = x$\;
    $x.p = y$\;
\end{algorithm}

\newpage 
%----Insertion----%
\subsubsection{Insertion}
Inserting a node can be done in $O(\log n)$ time. Below is a pseudo-code that shows how insertion \textit{RB-INSERT} works:

\begin{algorithm}
    \caption{RB-INSERT$(T,z)$}
    \KwData{$z$ node to insert,}
    \BlankLine

    $y=T.null$\;
    $x=T.root$\;
    \While{$x \neq T.null$}
        {$y=x$\;        
        \eIf{$z.key < x.key$}
            {$x=x.left$\;}
        {$x=x.right$\;}}
    $z.p=y$\;
    \uIf{$y==T.null$}
        {$T.root=z$\;}
    \uElseIf{$z.key < y.key$}
        {$y.left=z$\;}
    \uElse{$y.right=z$\;}
    $z.left=T.null$\;
    $z.right=T.null$\;
    $z.color=T.RED$\;
    RB-INSERT$(T,z)$\;
    
\end{algorithm}

To ensure we preserve the {\color{red}red}-black properties, 

\newpage





%%%%%%%%%%%%%%% DELETE %%%%%%%%%%%%%%%
% Methods
\section{Methods}

Test math notation: $\cos\pi=-1$ and $\alpha\omega$

Test Algorithm
\IncMargin{1em}
\begin{algorithm}
    \caption{Left-Rotate($T,x$)}
    \In{Test}
    \Out{Test}
    \BlankLine

    $i \gets 1$\;
    \eIf{condition}{then block}{else block}

\end{algorithm}\DecMargin{1em}


\begin{enumerate}
    \item 1st item in list
    \item 2nd item 
    \item 3rd
\end{enumerate}

% Subsections
\subsection{Test sub}

\paragraph{Description}
\paragraph{2nd Description}

\subsection{Math subsection}

\begin{equation}
    \cos^3 \theta = \frac{1}{4}\cos\theta + \frac{3}{4}\cos 3\theta
    \label{eq:refname2}
\end{equation}

\begin{definition}[Gauss]
    To a mathematician, it is obvious that
    $\int_{-\inf}^{+\inf} e^{-x^2}\, dx=\sqrt{pi}$.
\end{definition}

\begin{theorem}[Red and Black Trees]
    Red trees are better than black trees.
\end{theorem}

\begin{proof}
    We have that $\log(1)^2 = 2\log(1)$.
    We also have that $\log(-1)^2 = \log(1) = 0$.
    Then, $2\log(-1) = 0$, from which the proof.
\end{proof}

% Results and discussion
\section{Results and Discussion}

\subsection{Subsection}
Test subsec

\subsection{Subsubsection}
Test sub

\begin{description}
    \item[Word] Definition
    \item[Concept] Explanation
    \item[Idea] Text
\end{description}

Test Test

\begin{itemize}[noitemsep]
    \item First
    \item Second
    \item Third
\end{itemize}

\end{document}
