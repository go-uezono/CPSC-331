\documentclass[10pt, 
a4paper, 
oneside, 
headinclude, footinclude, 
BCOR5mm]
{scrartcl}

% Required packages
\usepackage[
    nochapters, beramono, eulermath, pdfspacing, dottedtoc,
]{classicthesis}
\usepackage{arsclassica}
\usepackage[T1]{fontenc}
\usepackage[utf8]{inputenc}
\usepackage{graphicx}
\graphicspath{}
\usepackage{enumitem}
\usepackage{lipsum}
\usepackage{subfig}
\usepackage{amsmath,amssymb,amsthm}
\usepackage{varioref}
\usepackage{titlesec}
\usepackage{color}
\usepackage[linesnumbered, lined, ruled, commentsnumbered]{algorithm2e}

% Adjust title size
\titleformat*{\section}{\color{RoyalBlue}\LARGE}
\titleformat*{\subsection}{\color{black}\Large}
\titleformat*{\subsubsection}{\color{gray}}
\titleformat*{\paragraph}{\large}

% Theorem Styles

% Style used for definitions and examples
\theoremstyle{definition}
\newtheorem*{definition}{Definition}

% Style used for theorems, lemmas, propositions, corollaries
\theoremstyle{plain}
\newtheorem{theorem}{Theorem}[section]
\newtheorem{lemma}{Lemma}

% Style used for remarks and notes
\theoremstyle{remark}

% Set up algorithm 
\SetAlgoLined
\SetKwData{Left}{left}\SetKwData{This}{this}\SetKwData{Up}{up}
\SetKwFunction{Union}{Union}\SetKwData{FindCompress}{FindCompress}
\SetKwInOut{In}{input}\SetKwInOut{Out}{output}
\newcommand\mycommfont[1]{\footnotesize\ttfamily\textcolor{purple}{#1}}

% Hyperlinks
\hypersetup{
colorlinks=true, breaklinks=true, bookmarksnumbered, urlcolor=webbrown, 
linkcolor=RoyalBlue, citecolor=webgreen,
pdfauthor={\textcopyright},
pdfsubject={},
pdfkeywords={},
pdfcreator={pdfLaTeX},
pdfproducer={LaTeX with hyperref and ClassicThesis}
}

\hyphenation{Fortran hy-phen-ation}

% Title and Authors
\title{\normalfont\spacedallcaps{CPSC 331: Data Structures, Algorithms, and their Analysis}}
\author{\spacedlowsmallcaps{Go Uezono}}

\begin{document}

% Headers
\renewcommand{\sectionmark}[1]{\markright{\spacedlowsmallcaps{#1}}}
\lehead{\mbox{\llap{\small\thepage\kern1em\color{halfgray} \vline}\color{halfgray}\hspace{0.5em}\rightmark\hfil}}

\pagestyle{scrheadings}

% Table of Contents
\maketitle
\setcounter{tocdepth}{2}
\tableofcontents
\listoffigures
\listoftables

\newpage
%Algorithmic Analysis%
\section{Algorithmic Analysis}
\subsection{Loop Invariants}
\subsection{Bound Functions}

%Data Structures%
\section{Data Structures}

%Lists%
\subsection{Lists}

%Stacks%
\subsection{Stacks}

%Queues%
\subsection{Queues}

%Binary Search%
\subsection{Binary Search Trees}
\newpage

% Red and Black Trees
\subsection{Red and Black Trees}

\begin{definition}  
\end{definition}
A concrete implementation of a \textbf{self-balancing} binary-search tree (reference here) that automatically maintains balance. 
Each node is assigned a color, hence the name. There are specific rules how to each node attains their color, which will be discussed here (reference here). 
\begin{itemize}
    \item Giving each node their respective color ensures that the tree maintains a certain balance.
\end{itemize}
After inserting an deleting nodes, complex algorithms are applied to check compliance with rules.
\begin{itemize}
     \item In case of deviations, to restore the prescribed properties by recoloring nodes and rotations.
\end{itemize} 

\subsubsection{Rotational Properties}

\begin{description}
    \item[Assumption] We will assume that values in the tree are unique. 
\end{description}
\newpage








%%%%%%%%%%%%%%% DELETE %%%%%%%%%%%%%%%
% Methods
\section{Methods}

Test math notation: $\cos\pi=-1$ and $\alpha\omega$

\begin{enumerate}
    \item 1st item in list
    \item 2nd item 
    \item 3rd
\end{enumerate}

% Subsections
\subsection{Test sub}

\paragraph{Description}
\paragraph{2nd Description}

\subsection{Math subsection}

\begin{equation}
    \cos^3 \theta = \frac{1}{4}\cos\theta + \frac{3}{4}\cos 3\theta
    \label{eq:refname2}
\end{equation}

\begin{definition}[Gauss]
    To a mathematician, it is obvious that
    $\int_{-\inf}^{+\inf} e^{-x^2}\, dx=\sqrt{pi}$.
\end{definition}

\begin{theorem}[Red and Black Trees]
    Red trees are better than black trees.
\end{theorem}

\begin{proof}
    We have that $\log(1)^2 = 2\log(1)$.
    We also have that $\log(-1)^2 = \log(1) = 0$.
    Then, $2\log(-1) = 0$, from which the proof.
\end{proof}

% Results and discussion
\section{Results and Discussion}

\subsection{Subsection}
Test subsec

\subsection{Subsubsection}
Test sub

\begin{description}
    \item[Word] Definition
    \item[Concept] Explanation
    \item[Idea] Text
\end{description}

Test Test

\begin{itemize}[noitemsep]
    \item First
    \item Second
    \item Third
\end{itemize}

\end{document}
